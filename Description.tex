\documentclass[12pt]{article}
\usepackage{pdfpages}
\usepackage[a4paper, left=15mm, right=10mm, top=20mm, bottom=20mm]{geometry}

% Поддержка русского языка
\usepackage[utf8]{inputenc}
\usepackage[russian]{babel}
\renewcommand{\familydefault}{\sfdefault}
\usepackage[T2A]{fontenc}
\usepackage{tikz} % Для рисования
\newcommand{\lt}{\ensuremath <}
\newcommand{\gt}{\ensuremath >}

\usepackage{amsmath} % Для псевдокода
\usepackage{amssymb}
\usepackage{mathtools}
\DeclarePairedDelimiter{\abs}{\lvert}{\rvert}
\usepackage{algorithm}
\usepackage{algpseudocode}
\usetikzlibrary{trees, positioning} % Подключаем библиотеку для работы с деревьями

\usepackage{array} % Для работы с таблицами
\usepackage{multirow}

\def\ojoin{\setbox0=\hbox{$\bowtie$} \rule[-.02ex]{.25em}{.4pt}\llap{\rule[\ht0]{.25em}{.4pt}}} % Для рисования внешних соединений
\def\leftouterjoin{\mathbin{\ojoin\mkern-5.8mu\bowtie}}
\def\rightouterjoin{\mathbin{\bowtie\mkern-5.8mu\ojoin}}
\def\fullouterjoin{\mathbin{\ojoin\mkern-5.8mu\bowtie\mkern-5.8mu\ojoin}}

\usepackage{listings} % Пакет для кода
\usepackage{xcolor}   % Цвета для подсветки синтаксиса
\lstset{
    language=C,                % Язык C
    basicstyle=\ttfamily,      % Моноширинный шрифт
    keywordstyle=\bfseries\color{blue}, % Ключевые слова (if, for) синие
    commentstyle=\itshape\color{gray},  % Комментарии серые
    stringstyle=\color{red},   % Строки красные
    numbers=left,              % Нумерация строк
    numberstyle=\tiny,         % Маленький шрифт номеров строк
    stepnumber=1,              % Номер каждой строки
    breaklines=true,           % Перенос длинных строк
    breakatwhitespace=false,   % Разрешить перенос внутри слов
    frame=single,              % Рамка вокруг кода
    keepspaces=true,            % Сохранение пробелов
    tabsize=2
}

\usepackage{hyperref} % Для кликабельных ссылок
\usepackage{graphicx}

\usepackage{enumitem}
\begin{document}
\begin{flushleft}
\section*{Описание Менеджера эвристик}
Из сложности DPsize \cite{Moerkotte}:
\begin{align*}
    I^{\text{chain}}_{\text{DPsize}}(n) &=
    \begin{cases}
        \frac{1}{48} (5n^4 + 6n^3 - 14n^2 - 12n), & n \text{ even} \\
        \frac{1}{48} (5n^4 + 6n^3 - 14n^2 - 6n + 11), & n \text{ odd}
    \end{cases}
    \\
    I^{\text{cycle}}_{\text{DPsize}}(n) &=
    \begin{cases}
        \frac{1}{4} (n^4 - n^3 - n^2), & n \text{ even} \\
        \frac{1}{4} (n^4 - n^3 - n^2 + n), & n \text{ odd}
    \end{cases}
    \\
    I^{\text{star}}_{\text{DPsize}}(n) &=
    \begin{cases}
        2^{2n-4} - \frac{1}{4} \binom{2n}{n-1} + q(n), & n \text{ even} \\
        2^{2n-4} - \frac{1}{4} \binom{2(n-1)}{(n-1)} + \frac{1}{4} \binom{(n-1)}{(n-1)/2} + q(n), & n \text{ odd}
    \end{cases}
    \\
    \text{with } q(n) &= n 2^{2n-1} - 5 \times 2^{n-3} + \frac{1}{2} (2^n - 5n + 4)
    \\
    I^{\text{clique}}_{\text{DPsize}}(n) &=
    \begin{cases}
        2^{2n-2} - 5 \times 2^{n-2} + \frac{1}{4} \binom{2n}{n} - \frac{1}{4} \binom{n}{n/2} + 1, & n \text{ even} \\
        2^{2n-2} - 5 \times 2^{n-2} + \frac{1}{4} \binom{2n}{n} + 1, & n \text{ odd}
    \end{cases}
\end{align*}

Заметим, что полные графы \text{clique} обладают высокой алгоритмической сложностью планирования
алгоритмом DPsize. Поэтому все полные графы от 4 и будем отдавать Geqo. Заметим, что также 
звёзды обладают высокой сложностью. Их можно разбить на цепи по центральной вершине.
Посмотрим на сложность планирования цепей и циклов: при количестве таблиц $\leqq 12$ сложность циклов не более чем в 2 раза 
больше цепей. Таким образом, при количестве таблиц $\leqq 12$ в цепи|цикле будем применять стандартный DP, инчае Geqo.

Пусть дан большой($\geqq 50$ таблиц) аналитический запрос с внутренними соединениями. Для его планирования будем выполнять 
следующие шаги:
\begin{enumerate}
    \item Представим запрос в виде связного графа, иначе выполним шаги 2 - 4 для каждой связной компоненты графа и соединим 
их декартовым произведение.
    \item Разобьём граф топологии: сначала выделим \textbf{clique} (с 4+ вершинами), затем циклы, найдём звёзды и разобьём их на цепи.
    \item Спланируем каждую топологию графа.
    \item Алгоритмом \textbf{GOO \cite[стр. 108]{Thomas} \cite{Allam}} соберём все планы топологий в один план.
    

%    1. Представим запрос в виде связного графа, иначе выполним шаги 2 - 4 для каждой связной компоненты графа и соединим 
%    ( проверка связности a и b делается как bool have_relevant_joinclause( PlannerInfo* root, RelOptInfo* a , RelOptInfo* b))
%    их декартовым произведение ( метод void make_rels_by_clauseless_joins(PlannerInfo* root, RelOptInfo* a, ListCell* b)).
%    2. Разобьём граф топологии: сначала выделим полные графы (от 4+ вершин), затем циклы, найдём звёзды и разобьём их на цепи.
    3.Спланируем каждую топологию графа, описал выше как это сделать, получим RelOptInfo* для каждой топологии.

\end{enumerate}
\centering \bibliographystyle{alpha}
\bibliography{Preambles/sources.bib}
\raggedright

\end{flushleft}
\end{document}
