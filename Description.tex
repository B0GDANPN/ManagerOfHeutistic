\documentclass[12pt]{article}
\usepackage{pdfpages}
\usepackage[a4paper, left=15mm, right=10mm, top=20mm, bottom=20mm]{geometry}

% Поддержка русского языка
\usepackage[utf8]{inputenc}
\usepackage[russian]{babel}
\renewcommand{\familydefault}{\sfdefault}
\usepackage[T2A]{fontenc}
\usepackage{tikz} % Для рисования
\newcommand{\lt}{\ensuremath <}
\newcommand{\gt}{\ensuremath >}

\usepackage{amsmath} % Для псевдокода
\usepackage{amssymb}
\usepackage{mathtools}
\DeclarePairedDelimiter{\abs}{\lvert}{\rvert}
\usepackage{algorithm}
\usepackage{algpseudocode}
\usetikzlibrary{trees, positioning} % Подключаем библиотеку для работы с деревьями

\usepackage{array} % Для работы с таблицами
\usepackage{multirow}

\def\ojoin{\setbox0=\hbox{$\bowtie$} \rule[-.02ex]{.25em}{.4pt}\llap{\rule[\ht0]{.25em}{.4pt}}} % Для рисования внешних соединений
\def\leftouterjoin{\mathbin{\ojoin\mkern-5.8mu\bowtie}}
\def\rightouterjoin{\mathbin{\bowtie\mkern-5.8mu\ojoin}}
\def\fullouterjoin{\mathbin{\ojoin\mkern-5.8mu\bowtie\mkern-5.8mu\ojoin}}

\usepackage{listings} % Пакет для кода
\usepackage{xcolor}   % Цвета для подсветки синтаксиса
\lstset{
    language=C,                % Язык C
    basicstyle=\ttfamily,      % Моноширинный шрифт
    keywordstyle=\bfseries\color{blue}, % Ключевые слова (if, for) синие
    commentstyle=\itshape\color{gray},  % Комментарии серые
    stringstyle=\color{red},   % Строки красные
    numbers=left,              % Нумерация строк
    numberstyle=\tiny,         % Маленький шрифт номеров строк
    stepnumber=1,              % Номер каждой строки
    breaklines=true,           % Перенос длинных строк
    breakatwhitespace=false,   % Разрешить перенос внутри слов
    frame=single,              % Рамка вокруг кода
    keepspaces=true,            % Сохранение пробелов
    tabsize=2
}

\usepackage{hyperref} % Для кликабельных ссылок
\usepackage{graphicx}

\usepackage{enumitem}
\begin{document}
\begin{flushleft}
\section*{Описание Менеджера эвристик}

Пусть дан большой аналитический запрос с внутренними соединениями. Для его планирования будем выполнять 
следующие шаги:
\begin{enumerate}
    \item Представим запрос в виде набора связных компонент, любые две таблицы из разных компонент не имеют условия соединения между собой. 
    \item На основе количества связных подграфов введём функцию бюджета в компоненте. Если его достаточно, то можно позволить более долгое время оптимизации
    и более оптимальные эвристики/алгоритмы.
    \item Для каждой связной компоненты будем выполнять итеративную декомпозицию на топологии и их планирование в цикле, пока не получим план для компоненты. 
    Если что-то получилось выделить, возвращаемся в начало цикла, если нет, то переходим на планирование выделенных топологий и снова в начало цикла:
    \begin{enumerate}
        \item Выделим из компоненты \textbf{плотные графы} (с 4+ вершинами) с плотностью от \textbf{density\_border}
        \item Выделим циклы.
        \item Если после выделения  плотных графов и циклов, компонента остаётся связной, то выделим дерево, иначе выделяем звёзды и цепи.
        \item Выделим звёзды с лучами длины до \textbf{ray\_length}.
        \item Выделим цепи.
    \end{enumerate}
    Звездой это группа вершин: центральная и связанные с ней цепи, каждые две цепи не связаны с друг другом в оставшейся топологии. 
    Центральная вершина имеет количество строк в 10+ раз больше чем у 2+ соседей или соседей от 3+.
    \newline
    Все выделенные топологии  отсортируем по количеству связных подграфов в топологии.
    Распределим бюджет компоненты по топологиям в соответствии со сложностями. Каждая топология после планирования становится вершиной, получаем новый граф.
    \newline
    Пусть \textbf{border} - это количество свзяных подграфов, больше которого топология считается сложной, меньше лёгкой.
    \newline
    Планирование:
    \begin{enumerate}
        \item Если бюджета для текущей топологии не осталось, то планируем эвристически, иначе смотрим на сложность. Для лёгкой топологий используем  встроенный ДП, для 
    тяжёлой эвристику. Эвристики:
        \begin{enumerate}
            \item Цепь: алгоритм GOO, и пока остаётся бюджет будем оптимизировать деревья размера до \textbf{k}.
            \item Цикл: найдем соединение с наибольшей кардинальностью, и разобьём цикл по этом соединению - уберём одну таблицу в этом соединении. Спланируем цепь, и присоединим удалённую таблицу.
            \item Звезда: сначала спланируем лучи как цепи, затем будем последовательно присоединять к центру использую GOO (по кардинальности) с центральной фиксированной вершиной.
            \item Плотный граф:  GOO с оптимизациями деревьев размера до \textbf{k}.
            \item Дерево: если достаточно бюджета, то используем IKKBZ/linDP, иначе GOO с оптимизациями деревьев размера до \textbf{k}.
        \end{enumerate}
    \end{enumerate}
    \item Объединим планы компоненты запроса алгоритмом GOO(по стоимости) с помощью декартового произведения.
\end{enumerate}


\end{flushleft}
\end{document}
